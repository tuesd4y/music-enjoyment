\documentclass[conference]{IEEEtran}
\usepackage{graphicx} 
\usepackage[utf8]{inputenc}
%\usepackage [naustrian] {babel}
% correct bad hyphenation here
\hyphenation{op-tical net-works semi-conduc-tor}


\begin{document}
	% paper title
\title{ Are Smartphone IMUs enough for assessing crowd music enjoyment?}

% author names and affiliations
\author{
    \IEEEauthorblockN{Tarek Hamid}
	\IEEEauthorblockA{	k11771398 / 521\\
	K11771398@students.jku.at}
		\and
    \IEEEauthorblockN{Christopher Stelzmüller}
	\IEEEauthorblockA{	k11814096 / 521\\
	K11814096@students.jku.at}
		\and
    \IEEEauthorblockN{Sebastian Tanzer}
	\IEEEauthorblockA{	k11826853 / 521\\
	K11826853@students.jku.at}
		\and
    \IEEEauthorblockN{Tobias Priewasser}
	\IEEEauthorblockA{	K11702409 / 521\\
	K11702409@students.jku.at}
		\and
}

% make the title area
\maketitle


\begin{abstract}
We aim to measure crowd engagement with the currently playing music in
order to create a continuous playlist with the enjoyable music. Playlists will be
generated using the spotify-API that can generate playlists based on multiple
seed parameters. The playlist starts with a predefined seed, that is then
adapted according to the crowd's response to already played song.
By measuring how the crowd interacts and dances to the music using the
accelerometer in the crowd's smartphones (by comparing the music BPM to
BPM of people dancing), multiple seed parameters (e.g. valence (more info on
what that is), tempo, or others) can be fine-tuned.
\end{abstract}
	
	\IEEEpeerreviewmaketitle

	\bibliographystyle{IEEEtran}
	\bibliography{References}


\end{document}